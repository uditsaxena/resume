\documentclass[letterpaper]{article}

\usepackage[T1]{fontenc}
\usepackage[margin=1in]{geometry}
\usepackage{lmodern}
\usepackage{hyperref}
\usepackage{microtype}
\usepackage{parskip}
\usepackage{titlesec}
\usepackage{fontawesome}
\usepackage{calc}
\usepackage{enumitem}
\usepackage{xcolor}
\usepackage{pagecolor}
\usepackage{enumitem}
%\usepackage{showframe}

\pagenumbering{gobble}
\setlist[itemize]{leftmargin=*}
\titlespacing{\section}{0pt}{2pt}{-3pt}
\hypersetup{colorlinks=true, linkcolor=black, urlcolor=blue}

\title{R\'esum\'e}
\author{Udit Saxena}

\begin{document}
\pagecolor{white}
\begin{center}
\huge
Udit Saxena\\
\small
\faEnvelope~\href{mailto:usaxena@umass.edu}{\nolinkurl{usaxena@umass.edu}}
\faPhone~\href{tel:14138245810}{\texttt{+1 (413) 824-5810}}
\faGlobe~\url{http://uditsaxena.github.io}
\normalsize
\end{center}

\section*{Education}
\vspace{-1mm}
\textbf{University of Massachusetts, Amherst}, MA \hfill Sept 2016 --
    Present\\
MS in Computer Science, College of Information and Computer Sciences
\vspace{-2mm}
\begin{description}[leftmargin=!, labelwidth=\widthof{Coursework },
        font=\normalfont]
    \item[Coursework:] Machine Learning,
                        Systems for Data Science
\end{description}
\vspace{-0.5mm}
\textbf{Birla Institute of Technology and Science}, Pilani, India \hfill
    Aug 2010 -- Aug 2015\\
BE (Hons.) in Computer Science,
MSc (Hons.) in Mathematics

\section*{Research Experience}
\textbf{Multivariate Time Series Analysis - Real Time Gesture Recognition},  \hfill
Aug 2014 - Dec 2014
\emph{Mentor: Prof. Navneet Goyal} \hfill
BITS Pilani\\
\vspace{-6mm}
\begin{itemize}
\item Built a model which accounts for the real time factor of most naturally occurring time series and is able to handle time series frame by frame, thereby recognizing an early stopping criterion for faster recognition. 
\vspace{-2mm}
\item Achieved a recognition rate of 93 percent on the AUSLAN (Australian Sign Language) Dataset across 2300 instances and a recognition rate of 91 percent on the Daily Sports and Activities Dataset across 9000 instances using 5 fold stratified cross validation.
\vspace{-2mm}
\item Currently working on a paper on the same.
\end{itemize}

\section*{Work Experience}
\textbf{Sprinklr}, Gurgaon, India \hfill July 2015 -- Aug 2016\\
\emph{Product Engineer, Core Team}\\
\vspace{-6mm}
\begin{itemize}
\item Worked on deploying and integrating large scale social media analytics systems to leverage brand owned media and earned media for data driven insights across more than 20 different social networks.
\vspace{-2mm}
\item Designed and developed a Single Sign On solution using OpenSAML for Sprinklr as an Identity Provider to manage customer authorization sessions across multiple Sprinklr product lines.
\vspace{-2mm}
\item Administered API integrations of enterprise solutions - SAP C4C, SAP Hybris and other third party social networks - to allow customers to bridge compatibility issues or overcome cost of migration. 
\vspace{-2mm}
\item Engineered REST-based API extensions to the core module and streamlined the core audit module.
\end{itemize}

\vspace{-0.6mm}
\textbf{MLPACK, Google Summer of Code} \hfill May 2014 -- Aug 2014\\
\emph{Intern, Core Contributor since May 2014}\\
\vspace{-6mm}
\begin{itemize}
\item Implemented Multi-Class Adaboost algorithms - Adaboost.M1, Adaboost.MH and the Adaboost.SAMME.
\vspace{-2mm}
\item Added weak learning algorithms - Decision Stumps using template based splitting, and Perceptrons (single layer neural networks) for the boosting algorithms suite.
\end{itemize}

\vspace{-0.5mm}
\textbf{Adobe Systems}, Bangalore, India \hfill Jan 2015 -- June 2015\\
\emph{Intern, Adobe Captivate}\\
\vspace{-6mm}
\begin{itemize}
\item Built the User Analytics feature to collect non-Personal Identity Information about Captivate users and setup a pipeline to clean, mine, analyze the data and provide insights for data driven decisions.
\end{itemize}

\section*{Skills}
\begin{itemize}
\item Java, C++, Python, MATLAB; Git, SVN; Linux; MongoDB, Elasticsearch, MySQL
\end{itemize}

\section*{Projects}
\textbf{Wikipedia bot:}
Wikipedia bot for vandalism detection. Given an edit of a Wikipedia article, the bot's task is to detect and flag ill-intentioned edits. Achieved an accuracy of 81 percent using SVM.\\
\textbf{Compiler:}
Designed a complete functional compiler in Python for a toy language as a part of the course Programming Languages and Compiler Construction.
\end{document}